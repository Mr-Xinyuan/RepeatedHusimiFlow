\documentclass[UTF8]{ctexart}

\usepackage{lmodern}
\usepackage{color,xcolor}
\usepackage{enumerate}
\usepackage{graphicx}

\usepackage{algorithm}
\usepackage{algorithmic}
\usepackage{amsmath}
\usepackage{braket}

\title{《多模态分析法》算法}
\author{邴泓霖}
\date{}


\begin{document}
\maketitle

{\large \heiti 我关于MMA算法中的每个步骤的分析和理解}。
\begin{enumerate}
%
\item {\heiti 原文}:A set of \textcolor{blue!90}{templates} on N wave vectors \{$\mathbf{k_j}$\} is 
                created for the wave functions $\psi=\mathrm{e}^{\mathrm{i}\mathbf{k}_{i}^{test}\cdot\mathbf{r}}$ 
                generated by the M wave vectors $\{\mathbf{k}_{i}^{test}\}$. Both sets of wave vectors lie along the 
                dispersion contour. Each \textcolor{blue!90}{template} is stored as a vector of values $u_i$ (length M) 
                and corresponds to eq. (12) along $k_j$.\\
    {\heiti 理解}:
    \begin{itemize}
    \item 第一句话的含义应该是用M个$\{\mathbf{k}_{i}^{test}\}$产生一个模板(templates)集\{$k_{i}^{test}$\},从最后一句话
        可以推断每个模板$\mathbf{u}_i$应该是个由\{$k_j$\}组成的向量集。
    \item 最后一句说$\mathbf{u}_i$的每个值对应于(12)式(Husimi函数)同时在$k_j$方向上。因此,我推断$\mathbf{u}_i = [ \mathbf{u}_i^{1}\ \mathbf{u}_i^{2}\ \cdots\ \mathbf{u}_i^{N}]$,
        其中$\mathbf{u}_i^{j}=\mathrm{Hu}(\psi=\mathrm{e}^{\mathrm{i}\mathbf{k}_{i}^{test}\cdot\mathbf{r}},k_j)$
    \item 对于第二句中提到的色散等值线,个人觉得应该是说$\mathbf{k}_{i}^{test}=\frac{\sqrt{2mE^{test}}}{\hbar}$和
            $\mathbf{k}_{j}=\frac{\sqrt{2m(E-V(\mathbf{r})}}{\hbar}$中的$E$和$E^{test}$位于同一条等值线上,即$E=E^{test}$
    \end{itemize}
%
\item {\heiti 原文}:The metric given by $d_i=\mathbf{v}\cdot\mathbf{u}_i$, where the vector $\mathbf{v}$ represents the \textcolor{red!80}{Husimi vector}, is computed for
                    each \textcolor{blue!90}{template} in step 1).\\
        {\heiti 理解}:
        \begin{itemize}
            \item Husimi矢量应该不是一个简单的矢量,而是一个矢量集\{$\mathbf{v}^{j}$\},即$\mathbf{v}=[\mathbf{v}^1\ \mathbf{v}^2\ \cdots\ \mathbf{v}^N]$
            \item 因此,我认为$d_i$应该不是一个标量而是一个$N\times N$张量,即$d_i^{j,j'}=\mathbf{v}^{j}\cdot\mathbf{u}_{i}^{j'}$\colorbox{red!60}{这点与原文中不同}
            \item 我觉得这一步是为测出$\psi_c$的波矢$\mathbf{k}$与测试平面波的波矢$\mathbf{k}_i^{test}$的相近程度。
        \end{itemize}
%
\item {\heiti 原文}:The maximum of the set \{$d_i$\} and the corresponding wave vector $\mathbf{k}_{i}^{test}$ are stored.\\
        {\heiti 理解}:
        \begin{itemize}
            \item 这一步应该是找出与$\psi_c$的波矢$\mathbf{k}$相近的测试平面波的波矢$\mathbf{k}_i^{test}$。
            \item 我个人认为应该将\{$d_i$\}集合中的所有列放在一起比较,可以加快收敛速度。(还没尝试)
                $$ d_i=  
                    \begin{bmatrix}
                        \mathbf{v}^{1}\cdot\mathbf{u}_i^{1}&\cdots&\mathbf{v}^{1}\cdot\mathbf{u}_i^{N}\\
                        \vdots & \ddots &\vdots\\
                        \mathbf{v}^{N}\cdot\mathbf{u}_i^{1}&\cdots&\mathbf{v}^{N}\cdot\mathbf{u}_i^{N}\\
                    \end{bmatrix}
                $$
            \item 我的操作是找出$\mathbf{d}_i$中元素的最大值的行${j_max}$与列${j'_max}$
        \end{itemize}
\item {\heiti 原文}:The contribution of the trajectory with wave vector $\mathbf{k}_{i}^{test}$ is determined by 
                    the vector $\mathbf{u}_i\frac{d_i}{\mathbf{u}_i\cdot\mathbf{u}_i}$\\
        {\heiti 理解}:
        \begin{itemize}
            \item 从这句话说重新加权的矢量$\mathbf{u}_i\frac{d_i}{\mathbf{u}_i\cdot\mathbf{u}_i}$由$\mathbf{k}_{i}^{test}$
                    决定。因为,$d_{i}^{j,j'}$的大小由$\mathbf{u}_{i}^{j'}$决定,而$\mathbf{u}_{i}^{j'}$又由$\mathbf{k}_{i}^{test}$决定。
            \item 测试的平面波的$\mathbf{k}_i^{test}$越接近$\psi_c$的$\mathbf{k}$,$d_i$的值越大。
        \end{itemize}
%
\item {\heiti 原文}:The re-weighted template vector is subtracted from the \textcolor{red!80}{Husimi vector}, i.e.$\mathbf{v}_i \rightarrow \mathbf{v}_i - \mathbf{u}_i\frac{d_i}{\mathbf{u}_i\cdot\mathbf{u}_i}$\\
        {\heiti 理解}:
        \begin{itemize}
            \item 当找出与$\mathbf{u}_{i}^{j'}$相匹配的$\mathbf{v}^{j}$后,应减去它们之间相同的部分,防止在下次迭代计算中误判。
            \item 计算操作可表达为$\mathbf{v'}^{j_{max}}=\mathbf{v}^{j_{max}}-\mathbf{u}_{i}^{j'_{max}}\frac{d_{i}^{j_{max},j'_{max}}}{\mathbf{u}_{i}^{j'_{max}}\cdot\mathbf{u}_{i}^{j'_{max}}}$
        \end{itemize}
\item {\heiti 原文}:All negative elements of $\mathbf{v}$ are set to zero.\\
        {\heiti 理解}:
        \begin{itemize}
            \item 我觉得这一步是为加快收敛速度,已判断出的$v^{j'_{max}}$不再被误判。
            \item 可能这句话另有含义,但在我想出的这套算法中没有体现。
        \end{itemize}
\item {\heiti 原文}:Steps 1)–6) are repeated until the metric $d_i$ dips
        below a threshold.\\
        {\heiti 理解}:
        \begin{itemize}
            \item 循坏停止的条件$\max{d_i}<eps$,
            \item 从这点说明$d_i\mathbf{k}_{i}^{test}$的方向应该是流的方向。
        \end{itemize}
\end{enumerate}
%
作者Douglas Mason在YouTube上留下的一条关于Huismi流的视频报告ID:\textbf{FZ7x5RxN\_4}。
\begin{algorithm}[ht] 
    \caption{多模态分析法(MAA:Multi-Modal Analysis)}
    \label{alg:maa}
    \begin{algorithmic}[1]
        \STATE 用M个波矢\{$\mathbf{k}^{test}_{i}$\}生成测试平面波$\psi=\mathrm{e}^{\mathrm{i}\mathbf{k}_i^{test}\cdot\mathbf{r}}$ ,将每个测试平面波用N个测试波矢\{$\mathbf{k}_j$\}(非测试平面波的波矢)生成模板集合的成员。
               两组波矢$\mathbf{k}_{i}^{test}$和$\mathbf{k}_{j}$都位于色散等值线上。每个模板(个数为M)存储的值$\mathbf{u}_i$中的每一个成员对应于Husimi函数值;
        \label{alg:maastep:create}
        \STATE 衡量标准由$\mathbf{d}_i=\mathbf{v}\cdot\mathbf{u}_i$给出,其中矢量$\mathbf{v}$表示Husimi矢量,$\mathbf{u}_i$表示在步骤\ref{alg:maastep:create})中的每个模板;
        \STATE 找出集合\{$d_i$\}的极大值和对应的波矢$\mathbf{k}^{test}_i$,并将它们保存下来;
        \STATE 带波矢$\mathbf{k}^{test}_{i}$的轨迹的贡献由重新加权的矢量决定$\mathbf{u}_{i}^{j'_{max}}\frac{\mathbf{d}_{i}^{j_{max},j'_{max}}}{\mathbf{u}_{i}^{j'_{max}}\cdot\mathbf{u}_{i}^{j'_{max}}}$;
        \STATE Husimi矢量减去加权后的模板矢量,即$$\mathbf{v}^{j_{max}}_{i}\rightarrow\mathbf{v}^{j_{max}}_{i}-\mathbf{u}^{j'_{max}}_{i}\frac{\mathbf{d}^{j_{max},j'_{max}}_{i}}{\mathbf{u}_{i}^{j'_{max}}\cdot\mathbf{u}_{i}^{j'_{max}}}$$
        \STATE 将$\mathbf{v}^{j_{max}}$的所有负元素设为零;
        \label {alg:maastep:loopend}
        \STATE 重复步骤\ref{alg:maastep:create})\,-\,\ref{alg:maastep:loopend})直到度量$\mathbf{d}_{i}^{j_{max},j'_{max}}$低于一个阈值;
        \STATE 矢量集合\{$d_{i}^{j_{max},j'_{max}}\,\mathbf{k}^{test}_i$\}用于近似处理后的Husimi流;
    \end{algorithmic}
\end{algorithm}
\end{document}